\chapter{Introduction}

Welcome to the early days of home computing.\\

Let me propose you a journey to the 80s of past century. During this decade, the computing industry experienced a big revolution. The lessons learned during the 70s talk about a future in which computers domain everything. A future in which computers are everywhere. Including our homes.\\

In these days, many companies produced home computers. This is the age of the Spectrum ZX, the Commodore 64, the Amstrad CPC, and of course the big family of MSX computers. Small machines with really limited capacity and resources had their place in many homes when we were just kids. We learned to play video-games with these early computers. We learned to code with them. We learned their potential. Their power to change the modern economies, to transform the World. And we learned all that was just the beginning.\\

Again, welcome to the early days of home computing. Welcome to Artemisa.

\section{What is Artemisa}

Artemisa is an 8-bit computer system that conforms the MSX-1 specification. There were many hardware vendors who produced MSX computers in the 80s. Sony, Canon, Casio, Sharp, Spectravideo, Panasonic, Philips, ... Artemisa is just another member of this big family, with some particularities.\\

One of these particularities is that Artemisa design began in 2018. That is 35 years after the initial design of the first MSX computers. But this is not new. In 2006, ESE Artists’ Factory designed the 1chipMSX. This is a FPGA-based computer system that implements the MSX-2 standard. In 2013, a Korean team of MSX enthusiasts known as Retroteam Neo created the Zemmix Neo as an evolution of the 1chipMSX design with inspiration from the classic Zemmix game console produced by Daewoo. Since then, many other teams along the globe had produced their own customizations of Zemmix Neo. Also, in 2020 a group of Spanish makers released the MSXVR. This is an enhanced MSX emulated by a Raspberry Pi with the physical appearance of a classic 8-bit computer. \\

So yes, Artemisa is just another member that joins the club of MSX computers designed in 21st century. But here comes another particularity. Artemisa is one of the few, along the Omega MSX, that do not use emulation techniques. Some other products as 1chipMSX or Zemmix Neo use FPGAs to emulate the whole MSX hardware. Some others as MSXVR or the Zemmix Mini uses software emulation on top of an ARM-based computer. In contrast, Artemisa (and Omega) use real integrated circuits and discrete logic. In other words, it is made using the same design techniques and materials used in the 80s, when the first MSX computers were also designed. For example, the heart of the Artemisa computer is a real Zilog Z80 microprocessor instead of a Z80 CPU reimplemented in an HDL language by reverse engineering and synthetized in an Altera Cyclone FPGA. \\

Please note this does not mean Artemisa is better (or worse) than its FPGA-based or software-based counterparts. It is likely that all the things you can do with Artemisa can also be done in any other MSX computer system, emulated or not. You will not observe any remarkable additional feature, more speed or more stability. All they are functional MSX computers you can enjoy. \\

\section{Why Artemisa?}

You might ask why you would need an Artemisa computer considering there are some other options out there. Good question! \\

One possible answer is that you might find Artemisa computer more authentic. Please note I am not saying it is more authentic. But just you might find so. There are some users that specially appreciate the real hardware. And they feel more comfortable using an MSX computer that uses the classic technologies, implemented with a real CPU, real VDP, real PSG, real memory, … A computer that is designed using the old-school techniques. If you are such a user, you will specially enjoy Artemisa. \\

In the case of Artemisa Homebrew DIY series, there is another reason for choosing it. You have the chance to learn how an MSX computer is designed from the top to the bottom. And I am not just saying you will have to assemble it. Thanks to this book, you will also have the chance to understand how every single chip works. You have access to the schematic designs of every single circuit and an educational guide about their operation. This is a complete lecture about the architecture and design of MSX computers. A book that will put in your hands the tools to understand how a microcomputer works under the hoods. If you are interested in computer architectures beyond the theory, you will find Artemisa was made for your needs. \\

In summary, the main reasons to choose Artemisa is because you feel the real hardware is particularly sexy and/or you want to learn how an MSX computer works. If you only want to have an MSX machine, probably other options are better. On one hand, a refurbished MSX or a new FPGA-based one are likely less expensive. On the other hand, some other homebrew designs such as the Omega are more powerful and MSX2-compliant, but they lack the extensive documentation to understand their design. The offer of Artemisa is educative. The goal is to make you learn computer architecture and design at the same time you build and assemble your own MSX computer. \\

\section{About this document}

As stated above, this handbook compiles all the technical knowledge about the Artemisa Homebrew DIY Computer series. You will find it useful to assemble the parts of the DIY kit. But also, to understand how an MSX computer works in detail. From the smaller chip to its most complex circuit. 

\section{Intended audience}

If you have purchased an Artemisa Homebrew DIY Computer series, you are likely an electronic hobbyist or an MSX fan and you feel skilled enough to assemble a computer system by your own. This means soldering the components following the detailed instructions you will find in this book. \\

The Artemisa model 101 is built using electronic components with large through-hole packages. Its integrated circuits are DIP (Dual Inline Package), with a pad separation of 2.54mm. Other components such as resistors and capacitors are also large ones. This resembles the design style of early 80s computers. But also lowers the entrance barrier for those that do not trust their own soldering skills. The grade of difficulty of assembling an Artemisa computer model 101 is low. \\

Thus, only a basic knowledge in soldering is required. If unsure, there are tons of guides and YouTube tutorials available online to learn from. Assembling this kind of electronic components is easy. Just forget your fears and enjoy the experience!\\

Also, if you want to take advantage of the educational opportunities of this book, some basic knowledge of computing and digital systems is required. The user is assumed to understand basic concepts such as bit, byte, memory address, bus, CPU, peripheral, logic gate, voltage level, etc. A degree in Computer Science is useful but not necessary.

\section{How to Read This Book}

There are some parts of the book that have a special meaning or must be read in some specific way. They are described in this section. 

\subsection{Theory of operation blocks}

These blocks will be represented as follows:\\

\begin{theory}[h!]{Theory of operation}
	Here goes a full detailed description of how a part of the circuit works. 	
\end{theory}

The theory of operation blocks will describe the theory behind a part of the system. A basic knowledge on computing and digital systems is required in order to fully understand these blocks. However, do not worry if you lack this knowledge or find it difficult to understand. This information is not essential to assemble the system. You can still build your Artemisa computer without fully understanding what these text blocks tell. This is just for educational purposes.

\subsection{Warning blocks}

Some operations require special attention to things that, when unnoticed, could cause damage in the circuit. In these cases, a warning text block will indicate you have to pay special attention, like this one:\\

\begin{warning}[Warning message!]
	This is a warning message that requires your attention.
\end{warning}

\subsection{Schematic diagrams}

Along this document, the different parts of the circuit are explained using schematic diagrams. They are not difficult to understand. But if they are totally unfamiliar to you, a brief introduction might be useful. \\

Most of the elements you will find in a schematic diagram are shown in Figure \ref{fig:schematic-elements}.\\

\begin{figure}[h]
	\centering
	\includegraphics[width=\textwidth]{figures/schematic-elements}
	\caption{Elements of a schematic diagram}
	\label{fig:schematic-elements}
\end{figure}

\begin{enumerate}
	\item Wires. This is the most building block of a schematic diagram. It connects two points electrically. This means there will be a copper path between them that will let the current flow. 
	\item Basic components. Elements that have one or more pins where wires can be connected. These are power supply terminals, capacitors, resistors, oscillators, logic gates, etc. The different symbols used in this manual are described in Table \ref{table:schematic-symbols}. 
	\item Integrated circuits. Components that contain a whole and complex circuit inside it. These are processors, memories, flip-flops, decoders, etc. 
	\item Connections from/to other schematic diagrams. The pointing edge indicates the direction of the signal. If it has two pointed edges, it is bidirectional. 
	\item Buses. Collections of parallel wires that transmit a multibit data. 
	\item Wire connections from/to buses. A label indicates what is the wire that is obtained from the bus. 
\end{enumerate}

\begin{table}[h!]
	\centering
	\begin{tabular}{ m{22mm}|l }	
		\centering\includegraphics{icons/cap-cer} & Ceramic capacitor\\
		\centering\includegraphics{icons/cap-elec} & Electrolytic capacitor\\
		\centering\includegraphics{icons/osci} & Clock oscillator\\
		\centering\includegraphics{icons/volt-5} & 5 volts power supply\\
		\centering\includegraphics{icons/volt-gnd} & Ground power supply\\
		\centering\includegraphics{icons/res} & Resistor\\
		\centering\includegraphics{icons/gate-and} & AND logic gate\\
		\centering\includegraphics{icons/gate-inv} & Inverter logic gate\\
	\end{tabular}
	\caption{Component symbols in schematic diagrams}
	\label{table:schematic-symbols}
\end{table}

If you still do not know what a bus, a capacitor or a logic gate is, do not panic. Most of these concepts will be elaborated in the chapter Artemisa Computer System.

\subsection{Number notations}

Some numerical values are expressed along this manual. The prefix formats shown in Table \ref{table:number-formats} are used to distinguish between different integer bases. 

\begin{table}[h!]
	\centering
	\begin{tabular}{ r|l }	
		{\bf Prefix} & {\bf Description}\\
		0b & Binary number format. Example: 0b01100011.\\
		0x & Hexadecimal number format. Example: 0x4A0129FE.\\
	\end{tabular}
	\caption{Number format prefixes}
	\label{table:number-formats}
\end{table}
